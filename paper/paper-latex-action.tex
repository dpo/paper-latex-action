\documentclass[review]{siamart190516}   % Use [final] when ready.

\usepackage{siampaper}
\usepackage{draftwatermark}
\SetWatermarkScale{8}
\SetWatermarkLightness{.97}

% For the Cahier du GERAD.
%\renewcommand{\year}{2020}     % If you don't want the current year.
\newcommand{\cahiernumber}{00}  % Insert your Cahier du GERAD number.

% For debugging.
%\usepackage{showframe}

% Meta-information for the PDF file generated.
\pdfinfo{/Author (Dominique Orban)
         /Title (Insert Title Here)
         /Keywords (keyword1, keyword2, keyword3)}

\title{
  Bounds on Eigenvalues of Matrices Arising from Interior-Point Methods%
}

\author{
  Chen Greif%
  \thanks{%
    Department of Computer Science,
    The University of British Columbia,
    Vancouver, BC, Canada.
    E-mail: \mailto{greif@cs.ubc.ca}.
    Research partially supported by an NSERC Discovery Grant.
  }
  \and
  Erin Moulding%
  \thanks{%
    Department of Mathematics,
    The University of British Columbia,
    Vancouver, BC, Canada.
    E-mail: \mailto{moulding@math.ubc.ca}.
    Research partially supported by an NSERC CGS M Scholarship.
  }
  \and
  Dominique Orban%
  \thanks{%
    GERAD and Department of Mathematics and Industrial Engineering,
    \'Ecole Polytechnique, Montr\'eal, QC, Canada.
    E-mail: \mailto{dominique.orban@gerad.ca}.
    Research partially supported by an NSERC Discovery Grant.
  }
}
\date{\today}

\begin{document}

  \linenumbers
  \maketitle

  \thispagestyle{firstpage}
  \pagestyle{myheadings}

  \begin{abstract}
    Insert abstract here. \smarttodo{Write abstract}
  \end{abstract}

  \begin{keywords}
    Keyword1, keyword2, keyword3
  \end{keywords}

  \begin{AMS}
    15A06,  % Linear equations
    65F10,  % Iterative methods for linear systems
    65F20,  % Overdetermined systems, pseudoinverses
    65F22,  % Ill-posedness, regularization
    65F25,  % Orthogonalization
    65F35,  % Matrix norms, conditioning, scaling
    65F50,  % Sparse matrices
    93E24   % Least squares and related methods
    90C06,  % Large-scale problems
    90C20,  % Quadratic programming
    90C30,  % Nonlinear programming
    90C51,  % Interior-point methods
    90C53,  % Methods of quasi-Newton type
    90C55   % Methods of successive quadratic programming type
  \end{AMS}

  \tableofcontents
  \listoftodos\relax

  \section{Introduction}

  This is the introduction. An optimization problem has the general form
  \begin{equation}
    \label{eq:opt-problem}
    \minimize{x \in \R^n} \ f(x) \quad \st \ \ell \leq c(x) \leq u.
  \end{equation}

  For the real stuff, check \cite{wright-orban-2002}.

  \section{Conclusion}

  Conclusions will come in due time. \smarttodo{Come up with witty conclusions}

  \bibliographystyle{abbrvnat}
  \bibliography{abbrv,\jobname}

\end{document}
